\documentclass[a4paper,11pt]{kth-mag}
\usepackage[T1]{fontenc}
\usepackage{textcomp}
\usepackage{lmodern}
\usepackage[latin1]{inputenc}
\usepackage[swedish,english]{babel}
\usepackage{modifications}
\title{Agile Development in a Solo Environment}

\subtitle{Duis autem vel eum iruire dolor in hendrerit in
          vulputate velit esse molestie consequat, vel illum
          dolore eu feugiat null}
\foreigntitle{Lorem ipsum dolor sit amet, sed diam nonummy nibh eui
              mod tincidunt ut laoreet dol}
\author{Mathias Lindblom}
\date{June 2015}
\blurb{Master's Thesis at KTH Royal Institute of Technology\\Computer Science and Communication (CSC)\\Supervisor: Sten Andersson\\Examiner: Olle B�lter}
 \trita{}
\begin{document}
\frontmatter
\pagestyle{empty}
\removepagenumbers
\maketitle
\selectlanguage{english}
\begin{abstract}
  This is a skeleton for KTH theses. More documentation
  regarding the KTH thesis class file can be found in
  the package documentation.

Lorem ipsum dolor sit amet, consectetuer adipiscing elit. Mauris
purus. Fusce tempor. Nulla facilisi. Sed at turpis. Phasellus eu
ipsum. Nam porttitor laoreet nulla. Phasellus massa massa, auctor
rutrum, vehicula ut, porttitor a, massa. Pellentesque fringilla. Duis
nibh risus, venenatis ac, tempor sed, vestibulum at, tellus. Class
aptent taciti sociosqu ad litora torquent per conubia nostra, per
inceptos hymenaeos.
\end{abstract}
\clearpage
\begin{foreignabstract}{swedish}
  Denna fil ger ett avhandlingsskelett.
  Mer information om \LaTeX-mallen finns i
  dokumentationen till paketet.

Lorem ipsum dolor sit amet, consectetuer adipiscing elit. Mauris
purus. Fusce tempor. Nulla facilisi. Sed at turpis. Phasellus eu
ipsum. Nam porttitor laoreet nulla. Phasellus massa massa, auctor
rutrum, vehicula ut, porttitor a, massa. Pellentesque fringilla. Duis
nibh risus, venenatis ac, tempor sed, vestibulum at, tellus. Class
aptent taciti sociosqu ad litora torquent per conubia nostra, per
inceptos hymenaeos.
\end{foreignabstract}
\clearpage
\tableofcontents*
\mainmatter
\pagestyle{newchap}


\chapter{Introduction}
This chapter will initially go through the problem statement and goals of the report. It will also show the motivation and estimated value of the entire report. The last section will outline the structure of the rest of the report.

\section{Problem Statement}
\noindent
\newline
The hypothesis being tested is:

\begin{itemize}
\item There are agile development methods and strategies suitable for small development groups consisting of as little as one person.
\end{itemize}

The question regarding how this hypothesis can be tested is quite hard. It is obvious that agile development strategies can be used in small groups or when working solo, hence the word \underline{\emph{suitable}}. However the word is ambiguous but it also should be given vagueness of the evaluation methods. It is important to realize that what is being evaluated is very complicated for explained reasons and to test the hypothesis one would have to conduct a rather large study, preferably both quantitative and qualitative, over a long period of time. In reality, the results of this thesis will give indications rather than answers and could be used in a research of larger scope that justifies the topic.

\section{Goals}
The main goal of the thesis is as follows:

\begin{itemize}
\item Produce an agile development plan or agile strategies suitable for a single developer up to small teams consisting of only a few people.
\end{itemize}
\noindent
Secondary goals:

\begin{itemize}
\item Theoretical comparison of popular agile development methods such as for example Scrum, XP, Kanban etc. 
\item Overall clarification of agile terms and thinking/reasoning. For example differentiate Agile from Lean.
\end{itemize}

\section{Motivation \& Value}
Agile software development can be seen as a trend in today's computer industry and the actual concept of the method changes every year, which is quite appropriate given what the methodology advocates. The method mainly targets advanced systems and encourage continuous improvement and rapid changes to new customer demands. The development process revolves around having collaboration between self-organizing and cross-functional teams with focus on communication together with a iterative development process. However the information available regarding how-to practically establish an agile development process in small companies with small projects is limited. The available literature usually give hints on where to start but focuses heavily on the full blown methods and leaves it up to the reader to decide what is practically possible for their limited sized development group or company.

\section{Thesis Report Outline}

\chapter{Background}


\section{The Development of Development Methodologies}


\section{Agile Development}

\subsection{The Agile Manifesto \& Principles}
The Agile Alliance is an organization formed by a group of computer industry experts that in 2001. They decided to come together in order to form general development values in order to improve the software development process for companies around the world \cite{key1} \cite{key2}. The result was The Agile Manifesto that states the following:

\begin{itemize}
\item \textbf{Individuals and interactions} over processes and tools.
\item \textbf{Working software} over comprehensive documentation.
\item \textbf{Customer collaboration} over contract negotiation.
\item \textbf{Responding to change} over following a plan.
\end{itemize}

Using the values stated by The Agile Manifesto, 12 principles were created that act as the characteristics for agile practices \cite{key1}. These are the principles:

\begin{enumerate}
\item Our highest priority is to satisfy the customer through early and continuous delivery of valuable software.
\item Welcome changing requirements, even late in development. Agile processes harness change for the customer's competitive advantage.
\item Deliver working software frequently, from a couple of weeks to a couple of months, with a preference to the shorter timescale.
\item Business people and developers must work together daily throughout the project.
\item Build projects around motivated individuals. Give them the environment and support they need, and trust them to get the job done.
\item The most efficient and effective method of conveying information to and within a development team is face-to-face conversation.
\item Working software is the primary measure of progress.
\item Agile processes promote sustainable development. The sponsors, developers, and users should be able to maintain a constant pace indefinitely.
\item Continuous attention to technical excellence and good design enhances agility.
\item Simplicity--the art of maximizing the amount of work not done--is essential.
\item The best architectures, requirements, and designs emerge from self-organizing teams.
\item At regular intervals, the team reflects on how to become more effective, then tunes and adjusts its behavior accordingly.
\end{enumerate}

The group that wrote The Agile Manifesto consisted of the following people: Kent Beck, Mike Beedle, Arie van Bennekum, Alistair Cockburn, Ward Cunningham, Martin Fowler, James Grenning, Jim Highsmith, Andrew Hunt, Ron Jeffries, Jon Kern, Brian Marick, Robert C. Martin, Steve Mellor, Ken Stewier, Jeff Sutherland, Dave Thomas \cite{key1} \cite{key2}.

\subsection{Misconceptions}

\section{Agile vs Waterfall}

\section{Agile vs Lean}

\section{Agile Methodologies}

\subsection{Scrum}

\subsection{Dynamic Systems Development Method (DSDM)}

\subsection{Extreme Programming (XP)}

\subsection{Kanban}

\subsection{Scrum ban}


\chapter{Method}
The method is divided in two main parts. The first section describes the theoretical evaluation of agile methodologies. The second part describes the project that has been developed using a few selected agile strategies.

\section{Theoretical Evaluation of Methodologies}

\subsection{Agile Methodology Point System} 

\section{Practical Evaluation of Agile Project}

\subsection{Project Description}
The actual project, to be developed using agile strategies, is a private initiative from Per-Arne Forsberg who has 25 years of successful international experience and knowledge in managing technically skilled units, products and projects around the world at Ericsson. Having been involved and engaged in change management, using the latest Lean and Agile methodology.
The project is developing a dynamic, scalable and robust interactive tutoring framework prototype for pre-academic students. Assisting students today according to the ?old school? is social but also ineffective since it is usually one to one communication or one to a few in a geographical location (you have to meet at pre-determined location at pre-determined times, like classrooms). The long term goal is to assist in communication between students and teachers (tutors willing to help out with their knowledge from their preferred location) in Sweden and in turn combat the negative (knowledge) grade curve that can be seen around the country (according to international studies, PISA). In short the idea is to create a interactive website that can be used for many to many type of communication between students and teachers. An exact specification of the project does not exist in order to encourage the agile development process.

\subsection{Chosen Methodology and/or Strategy}


\chapter{Results}


\section{Theoretical}

\subsection{Point System}

\section{Practical}

\subsection{Project}

\subsection{Developer \& Product Owner Experience}


\chapter{Discussion}


\section{Result Critique}

\section{Conclusions}

\subsection{Developers Thoughts}

\section{Future Work} 


\appendix
\addappheadtotoc
\chapter{RDF}\label{appA}

\begin{figure}[ht]
\begin{center}
And here is a figure
\caption{\small{Several statements describing the same resource.}}\label{RDF_4}
\end{center}
\end{figure}

that we refer to here: \ref{RDF_4}

\begin{thebibliography}{56}

\bibitem{key1}
Martin C. Robert, Martin Micah,
\emph{Agile Principles, Patterns, and Practices in C\#}.
Prentice Hall, Massachusetts,
2nd edition,
2006.

\bibitem{key2}
http://www.agilealliance.org

\end{thebibliography}

\end{document}
